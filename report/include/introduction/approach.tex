\section{The modern approach}

\todo{Talk about why CNNs are used more and more in drone racing, and basically
why it is the chosen approach, which leads to tackling the dataset generation
problem.}

As Deep Learning has known an exponential growth during the past decade,
computer vision applications tend to exploit the power of convolutional neural
networks more and more. Thanks to their impressive accuracy in specific problem
solving, CNNs are becoming the main choice for tasks such as: object detection,
object segmentation, object recognition, \emph{etc}\ldots

In drone racing, the latest works, which are often the best performing, tend to
employ CNNs for the sensing part, or even as an all-in-one solution (refer to
\todo{refer to litterature review}).\\


\section{The chosen approach}


The solution will be implemented iteratively, by gradually increasing the
complexity of the vision algorithms, such that progress can be made even if the
end goal is too difficult to achieve.\\

The project should attempt to fulfill the following goals:

\begin{itemize}
	\item{Recognize gates in the input image}
	\item{Detect and localize the closest gate's center}
	\item{Evaluate the closest gate's orientation and distance}
	\item{Plan a trajectory for the drone to fly across the gate center}
	\item{Refine the trajectory in real time, for an optimal flight time}
	\item{Make the drone follow that trajectory while adjusting in real time}
\end{itemize}
~\\
However, the following functionalities are out of the scope and will be
discarded:

\begin{itemize}
	\item{Detect any other obstacles in the input image}
	\item{Avoid obstacles that are on the drone's trajectory}
	\item{Map and localize the detected gates in world frame}
	\item{Plan a trajectory of a sequence of gates}
	\item{Apply online learning for adaptive racing}
\end{itemize}

