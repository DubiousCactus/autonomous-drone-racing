\section{Motivation}

\todo{Add abbreviations page!}

Each year, the robotics community gathers at conferences such as the
International Conference on Intelligent Robots and Systems (IROS) , or the
International Conference on Robotics and Automation (ICRA) where they exhibit
their discoveries and achievements in the field. Since the year 2014, IROS have
been the organizer and host of an annual autonomous drone race as part of its
conference. This competition brings experts in the field to focus their effort
on solving a very complex challenge that combines every aspect of robotic
systems: sensing, planning and acting. This yearly challenge provides an
entertaining opportunity for researchers to push the limits of autonomous
systems even further. It is an exciting case study that aims to motivate more
experts to develop innovative ways of solving complex problems, which are
applicable to other domains. Not only are they calling for breakthroughs in
autonomous drones, but also for all intelligent robotic systems and science in
general.

Inspired by this initiative, the Drone Racing League will be hosting a series
of races in which autonomous drones will compete and attempt to outperform a
professional FPV (First Person View) drone pilot.\\

\begin{figure}[h]
	\centering
	\includegraphics[width=0.5\textwidth]{figure/iros_2016.jpg}
	\caption{Autonomous drone racing "arena" at IROS 2016.}
	\label{fig:iros}
\end{figure}

Planned for the year 2019, the event is sponsered by the company \emph{Lockheed
Martin}, which will be offering a prize of 2 million USD for the winning
team~\cite{LockheedDRL}. Teams of university students and other drone
enthusiasts shall present innovative approaches on vision-based systems for
unmanned aerial vehicles: hence, they showcase their progress through racing.

\subsubsection{Prospections for autonomous drones}
The rise of consumer drones during the recent years indicates a craze for UAVs
(Unmanned Aerial Vehicle) from the average citizen. First seen as nothing more
but toys, the now well-known quadcopters are revealing their full potential,
thanks to the breakthroughs of robotics engineers and large businesses combining
their expertise to offer futuristic services.

Be it Amazon's yet to come \emph{Prime Air} service~\cite{PrimeAir}, promising
30 minutes deliveries by autonomous drones in a short radius of their
facilities, or even Ehang's electric sky-taxis driving a human passenger for up
to 10 miles~\cite{Ehang184}, more and more innovative companies are
demonstrating the possible applications of those agile fliers.  Indeed, it seems
this fantastic future is not far from our reach. \emph{Amazon Prime Air}
successfully delivered a package to its first customer in December
2016~\cite{PrimeAirFirst}, which took only 13 minutes and did not involve a
human pilot.\\

Moreover, reports from \emph{Goldman Sachs} \cite{TopTal} estimated that the
drone market would reach 100 billion USD between 2016 and 2020, referring to
the military, consumer and commercial sectors, as shown on
Figure~\ref{fig:toptal}.\\

\begin{figure}[h]
	\centering
	\includegraphics[width=0.7\textwidth]{figure/toptal.png}
	\caption{Drone market by sector: estimates for 2016-2020~\cite{TopTal}.}
	\label{fig:toptal}
\end{figure}

Based on those figures, it is safe to assume that autonomous drones hold a
promising future, and that active research in the field will not go to waste.
After all, autonomous drone racing might be a recreational segue to faster and
more precise self-flying vehicles for all industries.

\begin{figure}[h]
	\centering
	\includegraphics[width=0.7\textwidth]{figure/drone_market.png}
	\caption{Drone market by sector: estimates for 2016-2020~\cite{Goldman}.}
	\label{fig:goldmansachs}
\end{figure}
