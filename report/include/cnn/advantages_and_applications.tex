\subsection{Advantages and applications}

Neural networks have known a rapid increase in their performance in the last
decades, which allowed them to finally be used in real-world scenarios, such as
hand-written digits recognition, a solution used by banks and developed by Yann
LeCun \etal~\cite{41400} using a multi-layer perceptron.\\

However, most computer vision tasks, such as object detection or scene
recognition, are far above in terms of complexity. Indeed, the yearly
\emph{Large Scale Visual Recognition Challenge} organized by
\emph{ImageNet}~\cite{1409.0575} quickly pointed out the limitations of the
perceptron, and indirectly gave birth to the so-called \emph{convolutional
neural networks}.

This type of artificial neural networks attempts to mimic the visual cortex in
the mammal brain, by applying a long sequence of small filters to an input
image, in order to compute the latent space of an image: a dense representation
of the features with much fewer dimensions.

The power of a CNN over the traditional way of computing image features (e.g.
SIFT: \emph{Scale-Invariant Feature Transform}~\cite{SIFT, SIFT_2}) lies in its ability
to adapt to any objective through machine learning. Where a generic feature
detection algorithm would fail, a CNN would learn how to recognize the features
needed to minimize an objective function. It can be seen as a black box, where
the filters do not actually matter because they are computed and adjusted over
a dataset, as to feed the fully connected layers at the end of a CNN with the
relevant features.\\

To this day, convolutional neural networks are being used more and more, and
quite often without being acknowledged by the end user, in tasks ranging from
face recognition to pedestrian detection in autonomous vehicles.\\

In this work, the immense advantages of using a deep convolutional neural network
against a traditional geometry-based detection algorithm, are its robustness and
ability to adapt to change in color, shape, lighting conditions, as well as
overcoming motion blur and noise caused by fast movements of the camera.

Furthermore, a multi-layer perceptron can easily be connected to the last layer
of a CNN (as it is often the case), which leverages the learning ability and
offers the possibility to do more than just recognizing obstacles, but even
compute steering commands, as it was done in A. Loquercio et. al~\cite{dronet}.
