\subsection{Pretrained models}

The fast growing research in image recognition offers a broad choice of deep
learning models to be adopted for visual tasks. One can construct a
convolutional neural network model from the ground up, tailored to a specific
application, but it has far fewer chances to compete with a full fledged
general-purpose model developed by an entire team of researchers. This
reasoning, along with the lack of knowledge and experience in creating
efficient neural network architectures, is the reason why an already proven
model is to be used in this research.\\

However, because of the constraints imposed by the relatively poor
computational powers of the Intel Aero UAV, a small model with preferably few
parameters must be used, if on-board computation of the gate center detection is
to be achieved.\\

To begin with, the state-of-the-art models in visual recognition tasks were
eliminated, as they are out of scope regarding the light computation 
requirement, namely: ResNet-50 and its deeper variants, VGG-16, AlexNet,
Inception-v4, Inception-ResNet, \etc \ldots

The first choice is directed towards the model developed and used by Loquercio
\etal~in \emph{DroNet: Learn to Fly by Driving}~\cite{dronet}, mainly because
of its praised performance despite its low computational cost, but also because
its purpose, steering a drone in an urban environment, is in accordance with
the goals of this work.

As a second option, the state of the art in image recognition for lightweight
applications is used to compare the first model's performance on the same
dataset. Designed by Sandler \etal, \emph{MobileNetV2: Inverted Residuals and
Linear Bottlenecks}~\cite{MobileNetV2}, improves upon its previous iteration
and offers astounding results: in combination with SSDLite, it is $20\times$ more
efficient and $10\times$ smaller than YOLOv2 on the COCO dataset~\cite{MobileNetV2}.
