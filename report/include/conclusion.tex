\chapter{Conclusion}

\textcolor{red}{To be completed.\\}

Over the course of this project, a hypothesis on the use of synthetic images,
to solve the gate detection challenge in drone racing, was proposed and put
to the test. A complete semi-synthetic dataset generation pipeline was
implemented, and the realism of its product was analysed. In order to proceed
further with the evaluation of this method's viability, a simple yet efficient
flight controller was designed, based on a PID controller and a supervision
state machine.\\

To begin with, the synthetic gates generation method was tested on a first
model, DroNet. The results exposed severe symptoms of overfitting, and showed
the model's unability to learn a generalized objective function for this
problem.  Several regularization techniques were applied, but only delayed the
problem and did not actually solved it. In an attempt to bypass this impeding
issue, a more generalized model, MobileNetV2, was adopted. By using transfer
learning, different adjustments to the base topology of the network were tested
and compared against each other. Even though early signs of overfitting were
still present during the training of this model, the right regularization
techniques were applied, and the problem was considered solved.

Nevertheless, an even more crippling case of overfitting was encountered: the
model could categorize inputs to a great extent, but it was unfortunately
unable to detect any physical gates in real conditions. To overcome this
pitfall, a heavy image augmentation pipeline was set up, as to apply important
distortions reflecting the poor quality of the captured images from the drone's
camera.\\

In the end, the entire autonomous system developed during this work was tested
in real conditions. It was able to successfully cross several gates of
different shapes and colors, only one at a time. The result of those
experiments show the viability of the proposed method, and even though no
valuable test metrics are provided, a subjective evaluation of the system's
performance can be considered. This work shows promising results for drone
racing, and this innovative idea should empower researchers to more actively
develop control algorithms based on such simple, yet flexible, approach to
perception.
