\section{Gate detection}

To reiterate, the evaluation of the models' performance was mainly done by
judging its accuracy from a test video. To properly compare different
hyper parameters and fully connected layer configurations, as explained in
section~\ref{section:training}, several training sessions were done with a
relatively small dataset, until the best model topology is found.

\subsubsection{DroNet}

DroNet~\cite{dronet} is the first choice to be used for the perception. The
initial question is whether to use grayscale or RGB images. Intuitively,
grayscale images would be faster to compute but would also provide less
information for the network to extract features. Having painted the gates in
bright colors, it makes sense to train the network on RGB images so that it can
detect them better.

\begin{figure}[h]
	\centering
	\begin{tikzpicture}
	\pgfplotsset{
		scale only axis,
		legend pos=north east,
		legend entries={Grayscale,RGB},
		grid style={dotted,gray,line width=.1pt},
		grid=both
	}
	\begin{axis}[xlabel=Epoch,ylabel=Training loss]
		\addplot+[mark=none,smooth] table [y=train_loss,col sep=comma]
		{plots/dronet/dronet_gray_base_50000.txt};
		\addplot+[mark=none,smooth] table [y=train_loss,col sep=comma]
		{plots/dronet/dronet_rgb_base_50000.txt};
	\end{axis}
\end{tikzpicture}

	\caption{Grayscale and RGB training of DroNet with 50,000 samples.}
	\label{plot:grayscale-vs-rgb-dronet}
\end{figure}

Figure~\ref{plot:grayscale-vs-rgb-dronet} shows without a doubt that the
network is able to learn more from RGB input images, as the categorical
cross-entropy loss is decreased by a factor of 2 after only 50 iterations,
compared to grayscale input.

However, during the early experiments, the dataset only contained a few
textures (or colors) for the synthetic gates, which resulted in a very poor
performance of the RGB model in real world testing. The first intuition was
that a grayscale input would reduce the spectrum of colors to be identified,
which would be more robust to different colors of obstacles. In reality, even
though this idea turned out to offer promising results, the dataset was simply
lacking of variety in the color of the gates. Simply adding a lot more texture
files, to generate synthetic gates with more variation in color, solved the
issue and made the use of RGB inputs for the model possible and more
efficient.\\

To push the experiment further, the model is trained with RGB images only, and
different adjustments in the topology of the network are made. The following
results depicted in figure~\ref{plot:dronet-base-dropout-l2}, present different
divergences between the training and validation losses.

\begin{figure}[h]
	\centering
	\begin{subfigure}{0.49\textwidth}
		\begin{tikzpicture}
			\pgfplotsset{
				scale only axis,
				legend pos=north east,
				legend entries={Base,Dropout,L2 norm},
				grid style={dotted,gray,line width=.1pt},
				grid=both
			}
			\begin{axis}[xlabel=Epoch,ylabel=Training
				loss,width=7cm,height=7cm]
				\addplot+[mark=none,smooth] table [y=train_loss,col sep=comma]
				{plots/dronet/dronet_rgb_base_50000.txt};
				\addplot+[mark=none,smooth] table [y=train_loss,col sep=comma]
				{plots/dronet/dronet_rgb_hidden_dropout_50000.txt};
				\addplot+[mark=none,smooth] table [y=train_loss,col sep=comma]
				{plots/dronet/dronet_rgb_higher_l2_50000.txt};
			\end{axis}
		\end{tikzpicture}
		\caption{Training loss for 50,000 samples.}
	\end{subfigure}
	\begin{subfigure}{0.49\textwidth}
		\begin{tikzpicture}
			\pgfplotsset{
				scale only axis,
				legend pos=north east,
				legend entries={Base,Dropout,L2 norm},
				grid style={dotted,gray,line width=.1pt},
				grid=both
			}
			\begin{axis}[xlabel=Epoch,ylabel=Validation
				loss,width=7cm,height=7cm]
				\addplot+[mark=none,smooth] table [y=val_loss,col sep=comma]
				{plots/dronet/dronet_rgb_base_50000.txt};
				\addplot+[mark=none,smooth] table [y=val_loss,col sep=comma]
				{plots/dronet/dronet_rgb_hidden_dropout_50000.txt};
				\addplot+[mark=none,smooth] table [y=val_loss,col sep=comma]
				{plots/dronet/dronet_rgb_higher_l2_50000.txt};
			\end{axis}
		\end{tikzpicture}
		\caption{Validation loss for 50,000 samples.}
	\end{subfigure}

	\begin{subfigure}{0.49\textwidth}
		\begin{tikzpicture}
			\pgfplotsset{
				scale only axis,
				legend pos=south east,
				legend entries={Base,Dropout,L2 norm},
				grid style={dotted,gray,line width=.1pt},
				grid=both
			}
			\begin{axis}[xlabel=Epoch,ylabel=Training accuracy,width=7cm,height=7cm]
				\addplot+[mark=none,smooth] table [y=train_accuracy,col sep=comma]
				{plots/dronet/dronet_rgb_base_50000.txt};
				\addplot+[mark=none,smooth] table [y=train_accuracy,col sep=comma]
				{plots/dronet/dronet_rgb_hidden_dropout_50000.txt};
				\addplot+[mark=none,smooth] table [y=train_accuracy,col sep=comma]
				{plots/dronet/dronet_rgb_higher_l2_50000.txt};
			\end{axis}
		\end{tikzpicture}
		\caption{Training accuracy for 50,000 samples.}
	\end{subfigure}
	\begin{subfigure}{0.49\textwidth}
		\begin{tikzpicture}
			\pgfplotsset{
				scale only axis,
				legend pos=south east,
				legend entries={Base,Dropout,L2 norm},
				grid style={dotted,gray,line width=.1pt},
				grid=both
			}
			\begin{axis}[xlabel=Epoch,ylabel=Validation accuracy,width=7cm,height=7cm]
				\addplot+[mark=none,smooth] table [y=val_accuracy,col sep=comma]
				{plots/dronet/dronet_rgb_base_50000.txt};
				\addplot+[mark=none,smooth] table [y=val_accuracy,col sep=comma]
				{plots/dronet/dronet_rgb_hidden_dropout_50000.txt};
				\addplot+[mark=none,smooth] table [y=val_accuracy,col sep=comma]
				{plots/dronet/dronet_rgb_higher_l2_50000.txt};
			\end{axis}
		\end{tikzpicture}
		\caption{Validation accuracy for 50,000 samples.}
	\end{subfigure}
	\caption{Different regularization techniques in comparison with the
	baseline for DroNet.}
	\label{plot:dronet-base-dropout-l2}
\end{figure}


The most interesting point to notice is the obvious overfitting of the model
in the ``base'' configuration, where only the last fully connected layer has
been changed, as explained in section~\ref{section:dronet-mods}. This issue can
clearly be seen by the rising of the validation loss around epoch 25, while the
training loss for that model is still decreasing.

As an attempt to overcome this problem, two commonly used techniques of network
regularization are tested.

Firstly, a hidden layer of size 100 is added before the last fully connected
layer, along with a dropout probability of $0.5$. The original model disposes
of a dropout regularization of $0.5$ directly after the flattened output of the
last convolution layer, which is already quite heavy. To add more dropout, the
sensible way is to add one more hidden layer, which increases the capacity of
the network and therefore adds a risk of overfitting. However, the experiment
showed that the overfitting point was delayed to epoch 30 or so, and seems to
globally reduce that effect: the test is promising, but certainly not ideal.


As a second attempt, the L2 norm factor responsible for the regularization of
the network's weights was increased by a factor of 10. The results are clear:
the training is slower to converge to a minimal loss, since the weight updates
are constrained by the error receiving a penalty equal to the sum of the
squared weights, but there is no sign of overfitting during those 50
iterations. Seeing those conclusive results, a more in-depth experiment on the
benefits of a stronger L2 regularization is undertaken.\\

\begin{figure}[h]
	\centering
	\begin{subfigure}{0.49\textwidth}
		\begin{tikzpicture}
			\pgfplotsset{
				scale only axis,
				legend pos=north east,
				legend entries={50000 samples, 150000 samples},
				grid style={dotted,gray,line width=.1pt},
				grid=both
			}
			\begin{axis}[xlabel=Epoch,ylabel=Training loss,width=7cm,height=7cm]
				\addplot+[mark=none,smooth] table [y=train_loss,col sep=comma]
				{plots/dronet/dronet_rgb_higher_l2_50000.txt};
				\addplot+[mark=none,smooth] table [y=train_loss,col sep=comma]
				{plots/dronet/dronet_rgb_higher_l2_150000.txt};
			\end{axis}
		\end{tikzpicture}
		\caption{Training loss}
	\end{subfigure}
	\begin{subfigure}{0.49\textwidth}
		\begin{tikzpicture}
			\pgfplotsset{
				scale only axis,
				legend pos=north east,
				legend entries={50000 samples, 150000 samples},
				grid style={dotted,gray,line width=.1pt},
				grid=both
			}
			\begin{axis}[xlabel=Epoch,ylabel=Validation loss,width=7cm,height=7cm]
				\addplot+[mark=none,smooth] table [y=val_loss,col sep=comma]
				{plots/dronet/dronet_rgb_higher_l2_50000.txt};
				\addplot+[mark=none,smooth] table [y=val_loss,col sep=comma]
				{plots/dronet/dronet_rgb_higher_l2_150000.txt};
			\end{axis}
		\end{tikzpicture}
		\caption{Validation loss}
	\end{subfigure}
	\caption{Validation loss of DroNet in RGB with a higher L2 regularization
	factor.}
\end{figure}


What is clear from looking at figure~\ref{plot:dronet-l2}, is that not only is
the network still overfitting, even though it is happening after 50-60 epochs,
unlike the base model, but it is getting very sensitive to noise as it keeps
training. The heavy fluctuations in the validation loss and accuracy show the
symptoms of a model that is overfitting and predicting random classifications.

The fact that increasing the dataset size by a large amount, in addition to
trying different regularization techniques, did not seem to help prevent
overfitting strongly indicates that this particular model is unable to
learn a general mapping of the data.\\

As for the real world testing of this model, prediction videos generated from
test datasets showed that the validation accuracy was somewhat reflected on the
real gate detection: it performed well enough to fly the drone through
different types of physical gates, after a succession of trials.\\

\begin{table}[!h]
    \centering
    \caption{TODO: add real values}
    \begin{tabular}{llll}
        \toprule
		Dataset & Top 1 [\%] & Top 3 [\%] & Top 5 [\%]\\
        \midrule         
		Synthetic & \textbf{70.8} & \textbf{70.8} & \textbf{70.8} \\ 
        Real & \textbf{99.9} & \textbf{70.8} & \textbf{70.8} \\
        \bottomrule
    \end{tabular}
    \label{tab:my_label}
\end{table}

\subsubsection{MobileNetV2}

In the hope of achieving generalization and getting rid of overfitting
symptoms, the second model, MobileNetV2, is trained in the same conditions and
on the same dataset. Due to its nature, it is not possible to use grayscale
images for the input, which is not a problem in itself because it has been
deducted that RGB images provide more information for the network, and help
increase its classification performance.\\

To determine the optimal topology adjustments for the fully connected layer,
and therefore find the configuration yielding the best performance, a series of
training runs is compared and the results are shown on
figure~\ref{plot:mobilenetv2-base-comp}.

\begin{figure}[h]
	\centering
	\begin{subfigure}{0.49\textwidth}
		\begin{tikzpicture}
			\pgfplotsset{
				scale only axis,
				legend entries={Base,Dropout,Hidden dropout,Average pooling},
				grid style={dotted,gray,line width=.1pt},
				grid=both
			}
			\begin{axis}[xlabel=Epoch,ylabel=Training
				loss,width=7cm,height=7cm,legend
				style={at={(0.96,0.88)},anchor=north east}]
				\addplot+[mark=none,smooth] table [y=train_loss,col sep=comma]
				{plots/mobilenetv2/mobilenetv2_rgb_base_50000.txt};
				\addplot+[mark=none,smooth] table [y=train_loss,col sep=comma]
				{plots/mobilenetv2/mobilenetv2_rgb_dropout_50000.txt};
				\addplot+[mark=none,smooth] table [y=train_loss,col sep=comma]
				{plots/mobilenetv2/mobilenetv2_rgb_hidden_dropout_50000.txt};
				\addplot+[mark=none,smooth] table [y=train_loss,col sep=comma]
				{plots/mobilenetv2/mobilenetv2_rgb_avg_pooling_50000.txt};
			\end{axis}
		\end{tikzpicture}
		\caption{Training loss for 50,000 samples.}
		\label{plot:mobilenetv2-base-comp-train}
	\end{subfigure}
	\begin{subfigure}{0.49\textwidth}
		\begin{tikzpicture}
			\pgfplotsset{
				scale only axis,
				legend entries={Base,Dropout,Hidden dropout,Average pooling},
				grid style={dotted,gray,line width=.1pt},
				grid=both
			}
			\begin{axis}[xlabel=Epoch,ylabel=Validation
				loss,width=7cm,height=7cm,legend
				style={at={(0.96,0.88)},anchor=north east}]
				\addplot+[mark=none,smooth] table [y=val_loss,col sep=comma]
				{plots/mobilenetv2/mobilenetv2_rgb_base_50000.txt};
				\addplot+[mark=none,smooth] table [y=val_loss,col sep=comma]
				{plots/mobilenetv2/mobilenetv2_rgb_dropout_50000.txt};
				\addplot+[mark=none,smooth] table [y=val_loss,col sep=comma]
				{plots/mobilenetv2/mobilenetv2_rgb_hidden_dropout_50000.txt};
				\addplot+[mark=none,smooth] table [y=val_loss,col sep=comma]
				{plots/mobilenetv2/mobilenetv2_rgb_avg_pooling_50000.txt};
			\end{axis}
		\end{tikzpicture}
		\caption{Validation loss for 50,000 samples.}
		\label{plot:mobilenetv2-base-comp-val}
	\end{subfigure}
	\caption{Different topology adjustments in comparison with the baseling for
	MobileNetV2.}
	\label{plot:mobilenetv2-base-comp}
\end{figure}


At first glance, it is surprising to see such perfect straight line for the
training and validation losses of the model with dropout applied right before
the output layer. The probability was set to $0.5$, which is heavy but also a
standard value recommended by Srivastava~\etal~\cite{Dropout} in the original
paper presenting this regularization technique. The startling result is that
the model was completely unable to learn with such dropout.

However, adding one hidden layer of size 100 right before the output layer,
shows interesting results, but clearly performs worse than the baseline on the
training loss minimization (as shown on
figure~\ref{plot:mobilenetv2-base-comp-train}).  On the other hand, the base
model shows early signs of overfitting, as it can be observed that the
validation loss slowly rises around epoch 35, whereas it keeps decreasing for
the case where a hidden layer and dropout were added. The observation is that
this addition has potential benefits, but is slower to train; a longer training
session is required to draw final conclusions.


The last modification to the baseline is the use of average pooling for the
final convolution layer of MobileNetV2, since the base model does not perform
any sort of pooling. A direct gain in performance is observed, mainly on the
validation since it is slightly below the baseline. Unfortunately, it is
visible on figure~\ref{plot:mobilenetv2-base-comp-val}, that the model starts
to overfit around epoch 35, similarly to the base model.\\

To take the best of both worlds, one last experiment is performed where a
hidden layer of a hundred neurons, with a dropout of $0.5$, is added in
conjunction with average pooling on the last feature extraction layer. The
results can be seen on figure~\ref{plot:mobilenetv2-hdp}, and show that the
model converges around 80\% in accuracy, at a loss of 1, whereas it was
converging closer to a loss of 2 in the previous experiments. It is still
overfitting, but significantly less.

\begin{figure}[h]
	\centering
	\begin{subfigure}{0.49\textwidth}
		\begin{tikzpicture}
			\pgfplotsset{
				scale only axis,
				legend pos=north east,
				legend entries={Training,Validation},
				grid style={dotted,gray,line width=.1pt},
				grid=both
			}
			\begin{axis}[xlabel=Epoch,ylabel=Loss,width=7cm,height=7cm]
				\addplot+[mark=none,smooth] table [y=train_loss,col sep=comma]
				{plots/mobilenetv2/mobilenetv2_rgb_hidden_dropout_avg_pooling_noaug_150000.txt};
				\addplot+[mark=none,smooth] table [y=val_loss,col sep=comma]
				{plots/mobilenetv2/mobilenetv2_rgb_hidden_dropout_avg_pooling_noaug_150000.txt};
			\end{axis}
		\end{tikzpicture}
		\caption{Loss}
	\end{subfigure}
	\begin{subfigure}{0.49\textwidth}
		\begin{tikzpicture}
			\pgfplotsset{
				scale only axis,
				legend pos=south east,
				legend entries={Training,Validation},
				grid style={dotted,gray,line width=.1pt},
				grid=both
			}
			\begin{axis}[xlabel=Epoch,ylabel=Accuracy,width=7cm,height=7cm]
				\addplot+[mark=none,smooth] table [y=train_accuracy,col sep=comma]
				{plots/mobilenetv2/mobilenetv2_rgb_hidden_dropout_avg_pooling_noaug_150000.txt};
				\addplot+[mark=none,smooth] table [y=val_accuracy,col sep=comma]
				{plots/mobilenetv2/mobilenetv2_rgb_hidden_dropout_avg_pooling_noaug_150000.txt};
			\end{axis}
		\end{tikzpicture}
		\caption{Accuracy}
	\end{subfigure}
	\caption{RGB MobileNetV2 with 150,000 samples.}
	\label{plot:mobilenetv2-hdp}
\end{figure}


However, evaluating the model on test data revealed a severe overfitting on the
synthetic gates. Eventually, the trained model was unable to detect any real
gate, and all the performance gains brought by this model were starting to fade
away.\\

\begin{table}[!h]
    \centering
    \caption{TODO: add real values}
    \begin{tabular}{llll}
        \toprule
		Dataset & Top 1 [\%] & Top 3 [\%] & Top 5 [\%]\\
        \midrule         
		Synthetic & \textbf{70.8} & \textbf{70.8} & \textbf{70.8} \\ 
        Real & \textbf{99.9} & \textbf{70.8} & \textbf{70.8} \\
        \bottomrule
    \end{tabular}
    \label{tab:my_label}
\end{table}

\clearpage
