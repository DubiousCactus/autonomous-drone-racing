\subsection{Foreground blending}

The final step of forging semi-synthetic images is to merge the virtual scene
into a real environment, and to blend the two in a way that it looks as if it
were real.

First and foremost, a number of image deformations and distortions are applied
to the rendered virtual scene. This is needed because real cameras cannot
capture reality as it can be done in a controlled and sterile environment that
can provide computer graphics. In particular, the camera used on the UAV
produces a lot of noise, has a rather low resolution, does not perform well in
high dynamic range and is highly subject to motion blur.

If the generated image of a still and sharp gate is simply overlaid on top of
a blurry and noisy background, it will produce an incoherent and un-natural
result, triggering the signal for a ``fake'' image to anyone with a sense of
reason.\\

In order to apply a simulated motion blur, the amount of motion blur present in
the background image is first estimated. This is done via the Laplacian
operator, or the second order partial derivative, defined as

\begin{equation} \label{equ:laplacianoperator}
    \nabla f = \frac{\partial^2 f}{\partial x^2}
        + \frac{\partial^2 f}{\partial y^2},
\end{equation}

and which detects high changes in the pixel intensity, or in other words:
edges. A low amount of detected edges in the image reflects a blurry image,
where the pixel intensity is more evenly distributed. That is why the variance
of the Laplacian of the image is used with a set of three thresholds, found by
trial and error, to apply a different blur filter accordingly. Each of the
three blur kernels has different coefficients, meaning that a different amount
of synthetic motion blur is applied to the generated image by convolution,
depending on the background image. An example implemtentation of this
augmentation process can be seen in Fig.~\ref{fig:motionblur}.\\

Lastly, a reasonable amount of Gaussian-distributed additive noise is
incorporated to the generated image, so that it blends even better in the
background. The mean is left to its default value, 0, while the variance is
calculated from a configurable standard deviation, found by trial and error
until the result was satisfying (see Fig.~\ref{fig:addnoise} for reference).
