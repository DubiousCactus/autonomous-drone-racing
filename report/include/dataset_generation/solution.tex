\subsection{Proposed solution}

With the aim of providing a way for researchers to focus on the developement of
their solution, rather than the collection of a dataset, and to produce a
theoretically infinite amount of possible circuit configurations, a
semi-synthetic dataset generation is proposed, and further put to the test.\\

The idea proposed for the afformentioned solution is to generate a virtual
scene representing randomly positioned virtual gates that a drone would
potentially have to fly through in the real world. A rendered image of the said
scene is then overlaid onto a real image, taken from the drone's camera, as to
represent a random portion of a racing circuit, as seen from the drone's field
of view. 

\begin{figure}[h]
	\center
	\includegraphics[width=\textwidth]{figure/dataset_solution.png}
	\caption[The steps of a semi-synthetic image generation.]{A decomposition
	of the making of a semi-synthetic image, representing randomly positioned
	3D models of randomly selected gates overlaid on top of a randomly selected
	background image. The virtual scene's camera is positioned and oriented
	using the exact configuration of the background image, so that both
	perspectives match. The perspective grid is kept for visualization
	purposes.}
\end{figure}

In order to do so, several requirements have to be met.\\

Firstly, a motion capture system, such as the ones used in the movie production
industry, is used to record the position and orientation of the drone in 3D
space, also known as the extrinsic parameters of the camera. This is crucial
for the generation of a virtual scene whose perspective matches to the
perfection the perspective of the drone, so that the generated scene can later
be overlaid onto the real scene.

Secondly, the virtual camera must be a perfect model of the actual camera used
to capture the background images, since the perspective calculation is derived
from its intrinsic parameters. Therefore, the drone camera needs to be
calibrated in order to estimate its parameters, before being able to apply
them to the virtual camera.

Finally, the dimensions of the virtual scene must be in accordance with the
actual dimensions of the physical space where the dataset of background images
was recorded, otherwise unwanted artefacts such as gates being visible outside
the confinment of the real scene could be produced. The same reasoning can be
applied to the 3D models of gates or obstacles, since their scale most likely
matters for the usage of the generated imasges.\\

	\subsubsection{Advantages}
	\subsubsection{Limitations}
