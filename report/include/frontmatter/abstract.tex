% CREATED BY DAVID FRISK, 2016
Using synthetic images to train convolutional neural networks for autonomous
drone racing \\
Transfering knowledge from a semi-synthetic domain into the real world for drone
steering\\
Theo Morales\\
Department of Engineering\\
Aarhus University \setlength{\parskip}{0.5cm}

\thispagestyle{plain}			% Supress header 
\setlength{\parskip}{0pt plus 1.0pt}
\section*{Abstract}
The focus of this research will be on developing a robust and generalized
vision algorithm for detecting obstacles in a configurable manner and regardless
of their shape and color, such that the race gates can be specified prior to the
training of the algorithm, and thus tailored to the challenge itself. The end
goal is to greatly facilitate the learning of the vision-based detection model,
by automating the tedious and long process of collecting and annotating a large
dataset required to train the object detection algorithm using Deep Learning.

Indeed, this task is often the most time-consuming part of training
convolutional neural networks, mostly because each image must be annotated to be
used in supervised learning. By removing this weight off researchers' shoulders,
much more time is left to focus and experiment on the \emph{planning} and
\emph{acting} problems of the robotic paradigm \todo{reference to the diagram
above}.

In this work, the efficiency and robustness of tailored dataset generation for
Deep Learning-based computer vision algorithms, for the specific case of
autonomous drone racing, is analysed and discussed.
A simple approach to solving the \emph{planning} challenge is also undertaken,
which supports the praises of the proposed solution.

% KEYWORDS (MAXIMUM 10 WORDS)
\vfill
Keywords: drones, autonomous, dataset, deep learning, cnn.

%\newpage				% Create empty back of side
%\thispagestyle{empty}
%\mbox{}
