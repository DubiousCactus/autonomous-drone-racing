% CREATED BY DAVID FRISK, 2016
Synthetic images for convolutional neural networks in autonomous
drone racing \\
Transferring knowledge from a semi-synthetic domain into the real world for
trajectory planning\\
Théo Morales\\
Department of Engineering\\
Aarhus University \setlength{\parskip}{0.5cm}

\thispagestyle{plain}			% Suppress header 
\setlength{\parskip}{0pt plus 1.0pt}
\section*{Abstract}

In this work, the efficiency and robustness of tailored dataset generation for
deep learning-based computer vision algorithms, for the specific case of
autonomous drone racing, is analyzed and discussed. This particular challenge
is a perfect opportunity for researchers to stretch the limits of autonomous
robotic systems, since it combines the sensing, planning and acting phases into
one complex problem. The proposed solution focuses on the perception of self-flying
drones using state-of-the-art convolutional neural networks, and also attempts
to solve the sensing and planning problems as a proof-of-concept for the vision
algorithm. Effectively,
this research concentrates on developing a robust and generalized vision
algorithm for detecting obstacles regardless of their shape and color. In that
way, the racing gates can be specified prior to the training of the algorithm,
and thus tailored to the challenge itself. The end goal is to greatly
facilitate the learning of the vision-based detection model, by automating the
tedious and long process of collecting and annotating a large dataset required
to train the object detection algorithm using computer graphics. Indeed, this task
is often the most time-consuming part of training convolutional neural
networks, mostly because each image must be annotated to be used in supervised
learning. By removing this weight off researchers' shoulders, much more time is
left to focus and experiment on the \emph{planning} and \emph{acting} problems
of the robotics paradigm. A simple approach to solving the planning and acting
problems is also undertaken, which supports the praises of the proposed
solution. Finally, experiments are conducted and promising results are shown,
which promotes this original idea of using synthetic images to
train convolutional neural networks for drone racing. The method's advantages
are discussed, such as its faculty to generate an infinite amount of random
race circuit configurations, and therefore presenting its beneficial
scalability and flexibility. Furthermore, two image recognition models are
trained and compared using the generated dataset, in an attempt to leverage the
hypothesis made and show concrete results in real time conditions.
Naturally, the method's drawbacks, such as the limitations in data augmentation
and constraints of environment, are made clear and exposed for future
improvements.

% KEYWORDS (MAXIMUM 10 WORDS)
\vfill
Keywords: drone racing, deep learning, synthetic images, convolutional neural
network, autonomous vehicle.

%\newpage				% Create empty back of side
%\thispagestyle{empty}
%\mbox{}
