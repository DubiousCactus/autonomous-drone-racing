\chapter{Appendix 1}

\begin{figure}[h - 5em]
	\center
	\begin{python}
		view = Matrix44.look_at(
			# eye: position of the camera in world coordinates
			self.drone_pose.translation,
			# target: position in world coordinates where the camera is looking at
			self.drone_pose.translation + (self.drone_pose.orientation *
										   Vector3([1.0, 0.0, 0.0])),
			# up: up vector of the camera.
			self.drone_pose.orientation * Vector3([0.0, 0.0, 1.0])
		)
	\end{python}
	\caption[The \pyth{look_at()} function]{Creating the view matrix via the
	convenience \pyth{look_at()} function.}
        \label{fig:lookup}
\end{figure}
\begin{figure}[h!]
    \centering
    \begin{python}
        def add_noise(self, img):
            noisy_img = random_noise(img, mode='gaussian',
                                     var=self.noise_amount**2)
            return (255*noisy_img).astype(np.uint8)
    \end{python}
    \caption{Gaussian noise added to the generated image.}
    \label{fig:addnoise}
\end{figure}
\begin{figure}[h!]
    \centering
    \begin{python}
        def apply_motion_blur(self, img, amount=0.5):
            if self.no_blur: return cv_img
            if amount <= 0.3:
                size = 3
            elif amount <= 0.7:
                size = 5
            else:
                size = 9
            kernel = np.identity(size) / size
            return cv2.filter2D(img, -1, kernel)
    \end{python}
    \caption{Synthetic motion blur applied to the generated image.}
    \label{fig:motionblur}
\end{figure}
\begin{figure}[h!]
    \centering
    \begin{python}
reduce_lr = ReduceLROnPlateau(monitor="val_loss", factor=0.9, patience=5)
    \end{python}
    \caption{Keras' convenient ReduceLROnPlateau callback function.}
    \label{fig:reducelronplateau}
\end{figure}
